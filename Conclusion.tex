\chapter*{Conclusion}
%\pdfbookmark{Conclusion}{conclusion}
\addcontentsline{toc}{chapter}{Conclusion}

La d\'ecouverte r\'ecente d'un nouveau boson au LHC, compatible avec le boson de Higgs standard, a ouvert la voie \`a l'\'etude pr\'ecise du secteur \'electrofaible. En particulier, une \'etude fine du m\'ecanisme de brisure \'electrofaible est requise. Même s'il est  pour l'instant compatible avec le boson de Higgs du mod\`ele standard, ce nouveau boson pourrait tout aussi bien se r\'ev\'eler \^etre le boson de Higgs neutre le plus l\'eger pr\'edit par la supersym\'etrie. L'ILC, du fait de sa nature de collisionneur leptonique,  devrait fournir des mesures de pr\'ecision des propri\'et\'es et des couplages de ce nouveau boson. Des mesures d'une pr\'ecision d'un \`a deux ordres de grandeur meilleures que les pr\'ecisions actuelles devraient ainsi pouvoir \^etre atteintes. De telles pr\'ecisions permettront entre autres de réfuter ou d'affirmer certains mod\`eles de physique au delà du mod\`ele standard. Ces pr\'ecisions sont extrapol\'ees en fonction de l'environnement et des caract\'eristiques avantageuses d'un collisionneur $e^+ \, e^-$ lin\'eaire. Pour exploiter ces avantages, l'infrastructure de d\'etection devra s'adapter aux conditions de fonctionnement et fournir des pr\'ecisions encore in\'egal\'ees. Cela s'applique en particulier au d\'etecteur de vertex qui est en charge de l'identification des saveurs et qui participe \`a la reconstruction des traces de faibles impulsion. Par exemple, une identification des saveurs tr\`es performante est primordiale lorsque l'on veut mesurer les voies de d\'esint\'egrations hadroniques du boson de Higgs. 

\medskip

Des capteurs CMOS adapt\'es \`a la physique des hautes énergies sont d\'evelopp\'es depuis une quinzaine d'ann\'ees dans le groupe \textit{PICSEL} de Strasbourg. Du fait de leurs caract\'eristiques, ces capteurs sont de bons candidats pour le d\'etecteur de vertex de l'ILD. Ils poss\`edent en effet une granularit\'e importante menant \`a des r\'esolutions spatiales de l'ordre de quelques microm\`etres, tout en poss\'edant une efficacit\'e de d\'etection sup\'erieure \`a 99.5 $\%$, un taux d'impacts fant\^omes inf\'erieur \`a $10^{-5}$ (domin\'e par quelques pixels bruyants), et une vitesse de lecture ainsi qu'une r\'esistance aux radiations adapt\'ees aux conditions de l'ILC. Afin de composer avec les exigences sur la r\'esolution du param\`etre d'impact un d\'etecteur de vertex pour l'ILD compos\'e de capteurs CMOS a \'et\'e imagin\'e. Il s'agissait alors d'atteindre un budget de mati\`ere r\'eduit (de l'ordre de 0.3 $\% \, X0$) et une r\'esolution spatiale accrue (de l'ordre de 3 $\mu m$). Une g\'eom\'etrie compos\'ee de trois couches \'equip\'ees de capteurs sur leur deux faces a \'et\'e propos\'ee. Cette g\'eom\'etrie bas\'ee sur des couches doubles faces permet d'optimiser le budget de mati\`ere en n'utilisant qu'un seul support pour deux couches.

\medskip

Afin d'\'etudier les futurs d\'etecteurs de vertex, la projet europ\'een AIDA, par le biais du WorkPackage 9.3, offrira un nouveau télescope en faisceau compos\'e d'un premier bras de télescope de grande surface, suivi d'une cible et d'un secteur de détecteur de vertex. Un partie de ce premier bras de télescope pourra \^etre compos\'e de trois super-plans SALAT, et le secteur de d\'etecteur de vertex pourra \^etre constitu\'e d'échelles doubles faces \textit{PLUME}. Des tests en faisceaux de ces deux objets ont ainsi \'et\'e men\'es \`a bien afin de s'assurer de leur bon fonctionnement. Les r\'esultats de ces tests ont \'et\'e d\'elivr\'es dans cette th\`ese. Ils d\'emontrent le bon fonctionnement et l'homog\'en\'eit\'e de la r\'eponse en terme de r\'esolution spatiale, d'efficacit\'e de d\'etection, de taux d'impacts fant\^omes et de multiplicit\'e des amas, des capteurs composants CMOS ces deux objets. 

\medskip

Une simulation num\'erique de ces objets a ensuite \'et\'e impl\'ement\'ee en se basant sur la simulation du capteur MIMOSA-28. Comme de nombreuses inconnues sur les caract\'eristiques physiques de ce capteur existaient, une approche pragmatique a \'et\'e utilis\'ee. Ainsi, la r\'eponse du capteur a \'et\'e ajust\'ee \`a l'aide des donn\'ees recueillies lors des campagnes de tests en faisceau de MIMOSA-28 et MIMOSA-22. Une restitution avec une pr\'ecision sup\'erieure \`a 10 $\%$ de la r\'eponse du capteur a \'et\'e atteinte. Une fois la simulation de ce capteur r\'ealis\'ee, les \'echelles PLUME et les super-plans SALAT ont \`a leur tour \'et\'e simul\'es. Puis, grâce \`a ces simulations, des m\'ethodes d'alignement de ces objets dans leur globalit\'e, bas\'ees sur la minimisation d'un $\chi^2$ local, ont \'et\'e impl\'ement\'ees. Les caract\'eristiques des capteurs et des \'echelles \textit{PLUME} simul\'es ont alors \'et\'e \'etudi\'ees.

\medskip

Les r\'esolutions spatiales demand\'ees par le détecteur de vertex de l'ILD demandent un alignement de ce dernier avec une pr\'ecision sup\'erieure \`a la pr\'ecision intrins\`eque des capteurs utilis\'es. Une pr\'ecision sur l'alignement de l'ordre du microm\`etre est requise afin de ne pas d\'egrader les performances du détecteur. Outre l'obtention d'un meilleur budget de mati\`ere, l'utilisation de couches doubles faces de capteurs permet la reconstruction de mini-vecteurs entre les deux impacts caus\'es par le passage des particules charg\'ees \`a travers une double couche. Chaque mini-vecteur apporte alors une information suppl\'ementaire sur la direction de la trace traversant la double couche. On peut alors se questionner sur la valeur ajout\'ee de ces mini-vecteurs en terme de trajectom\'etrie, et en particulier en terme d'alignement. Dans cette th\`ese nous avons explor\'e le potentiel d'un alignement avec mini-vecteurs. L'alignement selon les zones de recouvrement des \'echelles se chevauchant \`a l'int\'erieur d'une couche est particuli\`erement int\'eressant puisqu'il permet \`a priori de s'affranchir de certains modes faibles obtenus lors d'une proc\'edure d'alignement globale. Il permet de plus de s'affranchir de la reconstruction compl\`ete des traces, puisque seuls les mini-vecteurs sont n\'ec\'essaires.

\medskip

Une \'etude de l'alignement avec mini-vecteurs selon la zone de recouvrement de deux \'echelles \textit{PLUME} simul\'ees a \'et\'e effectu\'ee. Une r\'esolution spatiale de 3.5 $\mu m$ (\`a incidence normale) sur chacune des faces des deux \'echelles \textit{PLUME} a \'et\'e utilis\'ee. Une r\'esolution diff\'erente est envisag\'ee pour les trois doubles couches du d\'etecteur de vertex de l'ILC. Cependant, la configuration utilis\'ee dans cette th\`ese permet de donner des ordres de grandeur sur ce qu'il sera possible de faire \`a l'ILC. La m\'ethode d'alignement utilis\'ee se base sur la minimisation d'un $\chi^2$ local. Un d\'esalignement jug\'e pessimiste de tous les degr\'es de libert\'e a \'et\'e utilis\'e. Puis une minimisation du $\chi^2$ a \'et\'e effectu\'ee. Notre m\'ethode d'alignement avec mini-vecteurs r\'ealis\'ee avec des objets Monte-Carlo parfaits, permet alors un alignement sub-microm\'etrique pour les coordonn\'ees du centre de l'\'echelle \`a aligner ; et restitue les inclinaisons de cette \'echelle avec une pr\'ecision meilleure que $10^{-2}$ mrad pour les rotations $O1Y$ et $O1Z$ et meilleure que $10^{-1}$ mrad pour la rotation $O1X$. Cette pr\'ecision est obtenue avec des particules de hautes impulsions ($120 \, GeV/c$), la g\'eom\'etrie de la double couche 2 (ou 3), une statistique de 10 000 paires de mini-vecteurs bien associ\'ees, un support de $2 \, mm$ entre les deux faces des \'echelles et avec une r\'esolution spatiale de $3.5 \, \mu m$ sur chacune des faces des \'echelles utilis\'ees.

\medskip

La m\'ethode utilis\'ee dans cette th\`ese donne un avant goût de ce qui est r\'ealisable en terme d'alignement gr\^ace aux mini-vecteurs. En effet, un alignement global de l'ensemble d'une couche, \`a l'aide de toutes les zones de recouvrement de cette couche, peut \^etre envisag\'e. Cet alignement devrait permettre de contraindre l'expansion radiale de chaque \'echelle puisqu'on pourra prendre en compte certaines contraintes relative \`a la couche consid\'er\'ee, comme par exemple, son rayon.

% Pour la configuration utilis\'ee lors de notre \'etude, tout les param\`etres d'alignement retrouv\'es apr\`es alignement sont comparables, aux incertitudes pr\`es, aux valeurs de l'alignement parfait (valeurs Monte-Carlo) \`a l'exception de la position $Z1$ du centre de l'\'echelle. Cette position est faiblement contrainte dans l'expression du $\chi^2$, il s'agit d'un mode faible induit par le type de faisceau utilis\'e. Toujours, pour la configuration utilis\'ee, les pr\'ecisions obtenues sont sous le microm\`etre pour les coordonn\'ees $X1$ et $Y1$ du centre de l'échelle align\'ee \`a partir de 1000 couples de mini-vecteurs bien associ\'es. Pour les trois inclinaisons, des pr\'ecisions sup\'erieures \`a $10^{-2}$ degr\'e ont \'et\'e trouv\'ees \`a partir de 1000 couples de mini-vecteurs bien associ\'es. Nous avons aussi constat\'e que la pr\'ecision de la m\'ethode augmente lorsque la distance inter-\'echelle diminue. La pr\'ecision s'am\'eliore aussi lorsque la diff\'erence d'inclinaisons entre les deux \'echelles diminue puisqu'il y a rapprochement des deux \'echelles dans la zone de recouvrement. Lorsque le nombre de couples de mini-vecteurs bien associ\'es augmente, la pr\'ecision augmente. Ainsi, avec la configuration utilis\'ee, on atteint une pr\'ecision de l'ordre de $0.2 \, \mu m$ sur les coordonn\'ees $X1$ et $Y1$ et une pr\'ecision de $5 \, 10^{-3}$ degr\'e pour les trois inclinaisons \`a partir de 20 000 couples de mini-vecteurs bien associ\'ees. Une saturation sur l'am\'elioration de la pr\'ecision commence \`a apparaître aux alentour de 20 000 couples de mini-vecteurs bien associ\'ees utilis\'es. On ajoutera que le mode faible sur la coordonn\'ee $Z1$ a pu \^etre r\'eduit en utilisant des mini-vecteurs dot\'es d'inclinaisons plus vari\'ees selon leur inclinaison $tiltX$. On insistera sur le fait que tout les r\'esultats donn\'es dans cette th\`ese sont obtenus \`a haute impulsion, avec une certaine configuration, et pour des r\'esolutions spatiales sur chacune des deux faces des \'echelles de 3.5 $\mu m$.

\medskip

  Les r\'esultats \'enoncés ci-dessus pour notre m\'ethode d'alignement sans champs magn\'etique sont des r\'esultats id\'eaux. En effet, nous avons utilis\'e des traces de haute impulsion, non impact\'ees par la diffusion multiple, et la reconstruction et l'association des mini-vecteurs utilis\'es pour l'alignement étaient parfaites puisque nous n'avons utilis\'e qu'une seule trace par \'ev\'enement. Nous  avons vu ensuite que la meilleure option pour obtenir une statistique suffisante \`a l'alignement consiste \`a utiliser le bruit de fond faisceau. La nature m\^eme des mini-vecteurs permet de s'affranchir de la reconstruction totale des traces. Ainsi, les traces de bruit de fond peuvent \^etre utilis\'ees pour l'alignement. 

\medskip

  Nous avons \'egalement \'etudi\'e le bruit de fond \`a l'ILC gr\^ace \`a la simulation de celui-ci durant un train complet. Les \'etudes effectu\'ees ont permis d'\'etablir la densit\'e moyenne d'impacts de bruit de fond par croisement de faisceau et par unit\'e de surface selon l'axe du faisceau et selon l'angle radial $\phi$ du d\'etecteur (en coordonn\'ees cylindriques). Nous avons alors \'etudi\'e les impulsions associ\'ees aux impacts de bruit de fond. Cela nous a permis d'estimer la limite basse des impulsions pour lesquelles une statistique sur la zone de recouvrement est rapidement atteignable. Pour l'alignement, plus cette limite est haute, meilleur sera l'alignement. Des limites basses de 200 $MeV/c$ pour les double couches 1 et 2 et de 100 $MeV/c$ pour la double couche 3 ont \'et\'e choisies. Les impacts poss\'edant des impulsions sup\'erieures ou \'egales \`a ces limites permettent d'obtenir une statistique de l'ordre de 10000 paires de mini-vecteurs dans une dur\'ee comprise entre quelques minutes et une heure. Cela permettra donc un alignement ou un r\'e-alignement des double couches du d\'etecteurs de vertex tr\`es rapide. Les angles d'incidence associ\'es aux impacts de bruit de fond ont alors \'et\'e calcul\'es. Ces angles ont ensuite permis d'estimer les taux d'occupation moyens des capteurs CMOS sur la totalit\'e des couches du d\'etecteur de vertex. L'\'etude des angles d'incidence nous a de plus permis d'identifier des impacts issus de particules rasant les \'echelles ($\theta \approx 90$ degr\'es). Les particules associ\'ees \`a ces impacts traversent alors la couche \'epitaxi\'ee sur une tr\`es grande distance, parfois \'egale \`a la taille du capteur. Une partie de ces particules, provient de l'ext\'erieur du d\'etecteur et sont r\'etrodifus\'ees depuis les calorim\`etres de faisceau. Ces particules sont responsables d'une part importante (parfois 50 $\%$) du taux d'occupation des capteurs de la premi\`ere double couche. Ainsi, il serait b\'en\'efique de dévier ou d'arrêter ces particules avant qu'elles atteignent le d\'etecteur de vertex.
  
  \medskip
  
  Les temps de lecture et les tailles de pixels associ\'es \`a chaque couche ont alors \'et\'e optimis\'es pour limiter le taux d'occupation des capteurs \`a une valeur de l'ordre du pourcent. Les nouvelles caract\'eristiques des capteurs en fonction de la couche \'etudi\'ee sont donn\'ees dans le tableau \ref{tab:new_param_sensors_perspectives}. D'un point de vue technologique, une couche \'epitaxi\'ee r\'eduite de l'ordre de 15 $\mu m$ est \`a privil\'egier. De plus, pour atteindre les temps de lecture de la premi\`ere double couche il faudra utiliser deux zones de lecture lisant 4 lignes simultan\'ement chacune.
  
  \begin{table}[h]
   \small
   \centering
    \begin{tabular}{|l|l|l|l|l|l|}
     \hline
     \textbf{Couche}            & \textbf{Pixels $\mu m^2$} & \textbf{R\'eso. $[\mu m]$} & \textbf{Lecture $[\mu s]$} & \textbf{Couche Epi $[\mu m]$} & \textbf{Sortie} \\ \hline
     \textbf{0}                 & $17 \times 17$  & 2.8                           & 8 (14 Bxs)                    & 15                            & Binaire         \\ \hline
     \textbf{1}                 & $17 \times 85$  & 5                             & 4 (7 Bxs)                     & 15                            & Binaire         \\ \hline
     $2 \rightarrow 5$ & $34 \times 34$  & 4                             & 80 (144 Bxs)                  & 15                            & ADC 3-4 bits    \\ \hline
    \end{tabular}
    \caption{Nouveaux param\`etres pour les capteurs du d\'etecteur de vertex.}
    \label{tab:new_param_sensors_perspectives}
   \end{table}
  
  Avec ces nouvelles dur\'ees de lecture, la densit\'e moyenne d'impacts sur les couches du d\'etecteur de vertex est d'environ 1, 0.3, 0.35 et 0.075 Impact/$mm^2$/Lecture pour les couches respectives 0 et 1 et pour les double couches respectives 2 et 3. Il faudra alors pouvoir identifier les impacts associ\'es aux particules d'impulsions sup\'erieures ou \'egales \`a 200 $MeV/c$ pour la premi\`ere double couche et sup\'erieures ou \'egales \`a 100 $MeV/c$ pour la seconde et la troisi\`eme double couche.
  
  \medskip
  
  Afin d'identifier ces impacts on pourra par exemple utiliser les propri\'et\'es des mini-vecteurs \`a savoir leur information sur la position et leur information sur l'inclinaison locale de la trace et r\'ealiser des coupures sur ces deux param\`etres. Comme nous s\'electionnons les particules de bruit de fond de plus haute impulsion produite proche du point d'intéraction, on pourra aussi se servir d'une coupure sur la position du vertex reconstruit. On y inclura alors l'erreur du \`a la perte d'\'energie et \`a la diffusion multiple. On pourra aussi associer \`a ces coupures d'autres coupures sur la forme des amas de pixels. En effet, la forme des amas nous renseigne sur l'inclinaison approximative de la trace au niveau de l'impact. Nous avons en effet observ\'e une corr\'elation entre l'angle d'incidence $\theta$ des particules d'impulsions sup\'erieure ou \'egale \`a 100 $MeV/c$ et la position de l'impact selon l'axe $U$ des \'echelles (axe du faisceau). Le champ magn\'etique sera aussi d'une aide importante puisqu'il courbe plus les particules de plus basses impulsions transverses que celles de plus hautes impulsions transverses. Des \'etudes sur ces aspects pourront \^etre r\'ealis\'ees \`a l'aide d'un simulation compl\`ete de l'ILD incluant le bruit de fond machine. Elle permettront une estimation plus r\'ealiste des densit\'es d'impacts au niveau des zones de recouvrement inter-\'echelle et détermineront si l'identification, la reconstruction et l'association en paires des mini-vecteurs issus des particules de plus haute impulsion du bruit de fond sont possibles. Enfin, si tel est le cas, les précisions de notre m\'ethode utilisant les particules de plus haute impulsion bruit de fond pourront \^etre d\'elivr\'ees. Il faudra alors ajuster des h\'elices subissant une diffusion multiple et une perte d'\'energie dans les couches en utilisant les informations des paires mini-vecteurs associ\'ees.

\medskip

Comme dans un d\'etecteur de vertex, les \'echelles de capteurs sont soumises \`a de nombreuses contraintes, les d\'eformations \'evoluent au cours du temps. Ainsi, des alignements fr\'equents sont n\'ecessaires afin de garder la pr\'ecision sur les param\`etres physiques reconstruits. Avec la m\'ethode pr\'esent\'ee ici, la faible dur\'ee d'acquisition pour obtenir un lot de traces suffisant \`a l'alignement pr\'ecis des \'echelles entre elles est un atout important puisqu'elle authorise un r\'e-alignement tr\`es rapide, \`a l'\'echelle de l'heure.

% La densit\'e d'impacts issue du bruit de fond faisceau-faisceau est tr\`es importante sur la premi\`ere couche de d\'etection. Cette densit\'e est de l'ordre de 500 impacts par zone de recouvrement, pour une zone de recouvrement de la premi\`ere double couche. Cette densit\'e diminue ensuite de plus en plus que l'on s'\'eloigne de la r\'egion d'interaction pour atteindre environ 25 impacts par zone de recouvrement de la troisi\`eme double couche. 

% \medskip

% Un nombre de 1 ou deux couples de mini-vecteurs bien reconstruits et bien associ\'es par temps de lecture d'une double couche, permet une statistique de l'ordre de 10 000 couples de mini-vecteurs bien reconstruits et associ\'es dans une dur\'ee de l'ordre de la minute \`a quelques minutes. Ainsi, il faudra pouvoir bien reconstruire et associer un ou deux couples de mini-vecteurs parmi quelques dizaines \`a quelques centaines. Etant donn\'ee la densit\'e d'impacts sur les chacune des deux faces des \'echelles double face, la reconstruction des mini-vecteurs dans la zone de recouvrement des deux \'echelles constitue un d\'efi majeur. On pourra alors prendre en compte les formes des amas de pixels pour reconstruire les mini-vecteurs. L'association des couples de mini-vecteurs sera aussi tr\`es ardue. Pour effectuer celle-ci, on pourra utiliser les propri\'et\'es des mini-vecteurs \`a savoir une information sur la position et une information sur l'inclinaison locale de la trace. On pourra par exemple proc\'eder \`a l'alignement de la couche $n$ la plus externe en premier puis projeter les traces issues des couples de mini-vecteurs obtenus apr\`es alignement sur la couche $n-1$, afin de d\'eterminer une fen\^etre de recherche. Pour toutes ces op\'erations l'id\'eal sera de s\'electionner les traces poss\'edant la plus haute impulsion parmi toutes les traces de bruit de fond dans le but de minimiser l'impact de la diffusion multiple. On ajoutera que des \'etudes portant sur l'\'epaisseur du support des \'echelles double face seront n\'ecessaires afin de composer avec la diffusion multiple et la rigidit\'e des \'echelles. En particulier la r\'esolution angulaire des mini-vecteurs pourra \^etre \'etudiée en d\'etail en fonction de cette \'epaisseur puisque celle-ci d\'etermine les performances de notre algorithme d'alignement et les performances en terme de reconstruction et d'association des mini-vecteurs. 
% 
% \medskip


% Comme nous l'avons vu, \`a l'ILC le bruit de fond faisceau-faisceau domine le taux d'occupation des capteurs des premi\`eres couches. Ainsi, en assumant un taux d'occupation de $1\%$ sur la premi\`ere couche, un nombre d'impact de l'ordre du millier \`a quelques milliers est enregistr\'e par temps de lecture. S'il l'on prend une zone de recouvrement faisant grossi\`erement $1/10$ des capteurs, on obtient environ une centaine d'impacts sur la zone de recouvrement. Comme nous l'avons vu, une sélection de seulement 5 mini-vecteurs par temps de lecture est suffisante pour assurer une statistique de l'ordre de 10 000 mini-vecteurs par minute. Étant donn\'ee l'information angulaire apport\'ee par les mini-vecteurs, des coupures drastiques pourront \^etre utilis\'ee. Ces coupures pourront \^etre associ\'ees \`a la projection des traces passant par les couches deux et trois. Pour la seconde et la troisi\`eme couche les estimations de densit\'e de bruit de fond r\'ecentes permettent d'estimer les coups selon les zones de recouvrement de la seconde couche \`a quelques dizaines et ceux sur les zones de recouvrement la troisi\`eme couche \`a une dizaine. La reconstruction et l'association sera alors abordable plus facilement dans le cas de ces deux couches. Ainsi, si la reconstruction et l'association d'environ cinq mini-vecteurs par temps de lecture est r\'ealisable, on obtiendrait une statistique d'environ 10 000 couples de mini-vecteurs bien associ\'es dans une dur\'ee de l'ordre de la minute pour aligner les \'echelles selon leurs zones de recouvrement. Comme nous l'avons vu cette statistique est suffisante pour un alignement de pr\'ecision.

\medskip

Les r\'esultats pr\'esent\'es dans cette th\`ese sont r\'ealis\'es \`a l'aide d'objets parfaits, ne pr\'esentant aucune d\'eformation. Les objets r\'eels sont eux d\'eform\'es. Afin de r\'esoudre ce probl\`eme, l'algorithme d'alignement peut \^etre adapt\'e pour r\'ealiser l'alignement de l'\'echelle par petites parties. Mais cette segmentation demande une statistique plus importante et est de surcro\^it moins pr\'ecise ("bras de levier" moins important). Une autre solution pourrait \^etre d'ajouter des degr\'es de libert\'e suppl\'ementaires correspondants aux d\'eformations de chaque \'echelle. C'est ce que fait par exemple l'exp\'erience CMS. Ces d\'eformations seraient ainsi prises en compte dans le calcul du $\chi^2$. Les d\'eformations de chaque \'echelle pourraient \^etre mesur\'ees avant leur utilisation dans le d\'etecteur de vertex afin de fournir une valeur de d\'epart pour le futur alignement. Des tests en faisceaux d'une configuration de deux \'echelles pr\'esentant une zone de recouvrement devraient permettre de d\'evelopper la m\'ethode dans une cadre r\'eel et non id\'eal. Ainsi, les pr\'ecisions r\'eelles de la m\'ethode pourront \^etre mieux estim\'ees. Ce tests seront possibles grâce au télescope en faisceau AIDA.

\medskip 
 
le travail pr\'esent\'e ici ne constitue qu'une premi\`ere \'etape dans l'\'etude de l'alignement avec mini-vecteurs. Afin de progresser sur l'utilit\'e et le bien fond\'e de cette m\'ethode d'alignement, celle-ci doit \^etre \'etudi\'ee et am\'elior\'ee dans un cadre r\'eel. C'est ce que permettra entre autre le t\'elescope en faisceau AIDA. De plus, des \'etudes dans le cadre d'une simulation compl\`ete de l'ILD seront n\'ecessaires afin d'\'evaluer si l'on peut utiliser ou non le bruit fond faisceau avec notre m\'ethode d'alignement. Cette simulation compl\`ete devra comporter :
  
  \medskip
  
  \renewcommand{\labelitemi}{$\bullet$}
  \begin{itemize}
   \item Le bruit de fond faisceau,
   \item Le champ magn\'etique de l'ILD, et le champ \textit{anti-DID},
   \item Une simulation des diff\'erents capteurs utilis\'es dans le d\'etecteur de vertex. Et en particulier, une bonne restitution de la r\'esolution spatiale de chaque couche en fonction de l'angle d'incidence (nouveaux digitiseurs).
  \end{itemize}
  
  \medskip
  
  Pour r\'ealiser cette simulation, on pourra utiliser la simulation compl\`ete de l'ILD d\'ejà existante et lui rajouter une g\'eom\'etrie de double couche poss\'edant des recouvrements entre \'echelles successives similaire \`a celle d\'evelopp\'ee dans cette th\`ese. Afin de simuler le bruit de fond faisceau, on pourra utiliser des g\'en\'erations du bruit de fond r\'ealisé gr\^ace au logiciel \textit{guinea-pig} disponibles sur la grille de calcul de l'ILC. Le code de reconstruction pourra aussi \^etre d\'evelopp\'e en utilisant le framework de l'ILC (\textit{Marlin}). Celui-ci incluant en particulier des outils de trajectom\'etrie avec champ magn\'etique activ\'e. Il faudra alors r\'ealiser une \'etude sur la forme et les r\'esolutions des amas de pixels en fonction des angles d'incidence sur les couches et une \'etude du \textit{pattern recognition} et de la reconstruction et de l'association des \textit{mini-vecteurs}. Ensuite, une trajectom\'etrie bas\'ee sur les h\'elices et les mini-vecteurs devra \^etre d\'evelopp\'ee dans le but d'\'etudier l'alignement sur les zones de recouvrement avec les particules du bruit de fond faisceau. Il s'agit l\`a d'un travail sur le long cours qui pourra \^etre d\'ebut\'e lors d'une nouvelle th\`ese.
  
  \medskip
  
  La trajectom\'etrie avec mini-vecteurs est un domaine vaste qui requiert de red\'efinir l'ensemble du domaine en terme de mini-vecteurs. Il faut en effet, savoir reconstruire les mini-vecteurs lors de l'\'etape du \textit{pattern recognition}. Il faudra ensuite savoir associer ces mini-vecteurs par groupe de 2 ou plus pour reconstruire des pseudo-traces (track finding), et enfin, il faudra savoir les utiliser pour l'alignement. Cette th\`ese repr\'esente un point de d\'epart pour l'exploration de l'alignement avec mini-vecteurs. Ainsi, la trajectom\'etrie avec mini-vecteurs semble prometteuse.
  
%   Ainsi, les pr\'ecisions d'une nouvelle m\'ethode d'alignement bas\'ee sur les mini-vecteurs ont \'et\'e \'etudi\'ees dans un cas id\'eal \`a haute impulsion. Une simulation compl\`ete du d\'etecteur de vertex devrait permettre d'\'etudier la création et l'association des mini-vecteurs issus du beamstrahlung et la pr\'ecision de notre nouvelle m\'ethode d'alignement dans un cadre plus r\'ealiste. Les densit\'es d'impacts et les taux de particules de plus hautes impulsions du bruit de fond sur les zones de recouvrement, extrapol\'es dans ce chapitre, pourront \^etre d\'efinis plus pr\'ecis\'ement. De plus, les pr\'ecisions sur les param\`etres d'alignement pourront \^etre red\'efinies dans le cas de l'ILD, en utilisant les traces du bruit de fond faisceau-faisceau et les r\'esolutions et temps de lecture envisag\'es sur chaque double couches d\'efinis dans cette th\`ese. Enfin, des tests en faisceau d'échelles double face possédant une zone de recouvrement pourront \^etre r\'ealis\'es gr\^ace au télescope en faisceau AIDA et devraient permettre de connaître la pr\'ecision r\'eellement atteignable avec cette m\'ethode d'alignement en fonction de l'impulsion des particules utilis\'ee.

% Enfin, le travail pr\'esent\'e ici ne constitue qu'une premi\`ere \'etape dans l'\'etude de l'alignement avec mini-vecteurs. Afin de progresser sur l'utilit\'e et le bien fond\'e de cette m\'ethode d'alignement, celle-ci doit \^etre \'etudi\'ee dans le cadre d'une simulation compl\`ete de l'ILD. Ces simulations devrait permettre de savoir si ce nouveau type d'alignement est r\'ealisable ou non dans un cadre plus r\'ealiste que nos simples \'etudes r\'ealis\'ees dans un cas id\'eal. Cette simulation compl\`ete devra comporter :
%   
%   \medskip
%   
%   \renewcommand{\labelitemi}{$\bullet$}
%   \begin{itemize}
%    \item le bruit de fond faisceau-faisceau,
%    \item le champs magn\'etique de l'ILD, et le champs anti-DID,
%    \item Une simulation du comportement des diff\'erents capteurs utilis\'es dans le d\'etecteur de vertex. Et en particulier, une bonne restitution de la r\'esolution spatiale et du temps de lecture de chaque couche.
%   \end{itemize}
%   
%   \medskip
%   
% \`A l'aide de ces simulations un travail important sur la reconstructionn et l'association des couples de mini-vecteurs devra \^etre effectu\'e. Le d\'efi r\'eside dans l'identification des mini-vecteurs issus des traces de plus haute impulsion du bruit fond et dans leur association.
%   
%   \medskip
%   
% L'alignement avec mini-vecteur est un domaine vaste qui requiert de red\'efinir l'ensemble des domaines de la trajectom\'etrie en terme de mini-vecteurs. Il faut en effet, savoir reconstruire les mini-vecteurs (pattern recognition), savoir les associer par groupe de 2 ou plus pour reconstruire des traces (track finding), et enfin savoir les utiliser pour l'alignement. Ainsi, la trajectom\'etrie avec mini-vecteur semble prometteuse et pourra elle aussi \^etre \'etudi\'ee. La pr\'esente th\`ese repr\'esente un point de d\'epart dans l'exploration de l'alignement avec mini-vecteurs.
%   
% AIDA \\
% 
% D\'ecouverte du Higgs et pas de SuSY visible au LHC pour l'instant \\
% ILC collisionneur lin\'eaires = un des collisionneurs envisag\'e pour le futur. \\
% Physique \`a l'ILC \\
% D\'etecteur de vertex Perfs \\
% N\'ecessité alignement : nouvelles m\'ethode \\
% Test en faisceau PLUME et SALAT : vrai donn\'ees. \\
% Simu : pour \'etude d'alignement \\
% Validation Simu \\
% M\'ethode d'alignement avec mini-vecteurs \\
% Pr\'ecision m\'ethode dans cadre id\'eal \\
% Calculs rapides de la stat : encouragent pour la suite \\
% Perspectives : simu ILD compl\`ete, et vrai donn\'ees AIDA \\